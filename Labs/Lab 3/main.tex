%%%%%%%%%%%%%%%%%%%%%%%%%%%%%%%%%%%%%%%%%
%
% CMPT 432
% Fall 2019
% Lab 2
%
%%%%%%%%%%%%%%%%%%%%%%%%%%%%%%%%%%%%%%%%%

%%%%%%%%%%%%%%%%%%%%%%%%%%%%%%%%%%%%%%%%%
% Short Sectioned Assignment
% LaTeX Template
% Version 1.0 (5/5/12)
%
% This template has been downloaded from: http://www.LaTeXTemplates.com
% Original author: % Frits Wenneker (http://www.howtotex.com)
% License: CC BY-NC-SA 3.0 (http://creativecommons.org/licenses/by-nc-sa/3.0/)
% Modified by Alan G. Labouseur  - alan@labouseur.com
% Further Modified by Timothy M. Polizzi - Timpolizzi2@gmail.com
% Code Listings by LaTeX
%
%%%%%%%%%%%%%%%%%%%%%%%%%%%%%%%%%%%%%%%%%

%----------------------------------------------------------------
%	PACKAGES AND OTHER DOCUMENT CONFIGURATIONS
%----------------------------------------------------------------

\documentclass[letterpaper, 10pt]{article}

\usepackage[english]{babel} % English language/hyphenation
\usepackage{graphicx}
\usepackage[lined,linesnumbered,commentsnumbered]{algorithm2e}
\usepackage{listings}
\usepackage{fancyhdr} % Custom headers and footers
\pagestyle{fancyplain} % Makes all pages in the document conform to the custom headers and footers
\usepackage{lastpage}
\usepackage{url}
\usepackage{listings}
\usepackage{color}
\usepackage{qtree}
\usepackage{tikz}
\usetikzlibrary{automata, positioning, arrows}
\tikzset{node distance=2cm, every state/.style={semithick, fill=gray!10},initial text={},double distance=2pt,every edge/.style={draw,->,auto, semithick}}
\let\epsilon\varepsilon

\definecolor{codegreen}{rgb}{0,0.6,0}
\definecolor{codegray}{rgb}{0.5,0.5,0.5}
\definecolor{codepurple}{rgb}{0.58,0,0.82}
\definecolor{backcolour}{rgb}{0.95,0.95,0.92}

\lstdefinestyle{mystyle}{
    backgroundcolor=\color{backcolour},   
    commentstyle=\color{codegreen},
    keywordstyle=\color{magenta},
    numberstyle=\tiny\color{codegray},
    stringstyle=\color{codepurple},
    basicstyle=\footnotesize,
    breakatwhitespace=false,         
    breaklines=true,                 
    captionpos=b,                    
    keepspaces=true,                 
    numbers=left,                    
    numbersep=5pt,                  
    showspaces=false,                
    showstringspaces=false,
    showtabs=false,                  
    tabsize=2
}

\lstset{style=mystyle}

\fancyhead{} % No page header - if you want one, create it in the same way as the footers below
\fancyfoot[L]{} % Empty left footer
\fancyfoot[C]{page \thepage\ of \pageref{LastPage}} % Page numbering for center footer
\fancyfoot[R]{}

\renewcommand{\headrulewidth}{0pt} % Remove header underlines
\renewcommand{\footrulewidth}{0pt} % Remove footer underlines
\setlength{\headheight}{5pt} % Customize the height of the header

%----------------------------------------------------------------
%	TITLE SECTION
%----------------------------------------------------------------

\newcommand{\horrule}[1]{\rule{\linewidth}{#1}} % Create horizontal rule command with 1 argument of height

\title{	
   \normalfont \normalsize 
   \textsc{CMPT 432 - Fall 2019 - Dr. Labouseur} \\[10pt] % Header stuff.
   \horrule{0.5pt} \\[0.25cm] 	% Top horizontal rule
   \huge Lab Three\\     	    % Assignment title
   \horrule{0.5pt} \\[0.25cm] 	% Bottom horizontal rule
}

\author{Timothy Polizzi \\ \normalsize Timothy.Polizzi1@Marist.edu}

\date{\normalsize\today} 	% Today's date.

\begin{document}

\maketitle % Print the title

%----------------------------------------------------------------
%   CONTENT SECTION
%----------------------------------------------------------------

\noindent

\section{\textit{Crafting a Compiler}}

\begin{enumerate}

    \item 4.7 \\
    \begin{enumerate}
    
        \item Left \\
        \begin{tabular}{ccl}
        Start & $\Rightarrow_{lm}$ & E \$ \\
        & $\Rightarrow_{lm}$ & T plus E \$ \\
        & $\Rightarrow_{lm}$ & F plus T \$ \\
        & $\Rightarrow_{lm}$ & num plus F times F \$ \\
        & $\Rightarrow_{lm}$ & num plus num times (E) \$ \\
        & $\Rightarrow_{lm}$ & num plus num times (T plus E) \$ \\
        & $\Rightarrow_{lm}$ & num plus num times (F plus T) \$ \\
        & $\Rightarrow_{lm}$ & num plus num times (num plus F) \$ \\
        & $\Rightarrow_{lm}$ & num plus num times (num plus num) \$ \\
        \end{tabular}
        
        \item Right \\
        \begin{tabular}{ccl}
        Start & $\Rightarrow_{rm}$ & E \$ \\
        & $\Rightarrow_{rm}$ & T plus E \$ \\
        & $\Rightarrow_{rm}$ & T plus T \$ \\
        & $\Rightarrow_{rm}$ & T plus T times F \$ \\
        & $\Rightarrow_{rm}$ & T plus F times F \$ \\
        & $\Rightarrow_{rm}$ & T plus num times num \$ \\
        & $\Rightarrow_{rm}$ & T times F plus num times num \$ \\
        & $\Rightarrow_{rm}$ & F times num plus num times num \$ \\
        & $\Rightarrow_{rm}$ & num times num plus num times num \$ \\
        \end{tabular}
        
        
        \item precedence \\
        This grammar structures expressions by having order of operations enforced when the grammar is used in a leftmost derivation. Otherwise it will just parse the information.
        
    \end{enumerate}
    
    \item 5.2c \\
    \begin{lstlisting}
procedure Start(tokenStream)
	switch(tokenStream.peek())
		case {num, lparen}
			call Value()
			match($)
	end

procedure Value(tokenStream)
	switch(tokenStream.peek())
		case {num}
			match(num)
		case {lparen}
			match(lparen)
			call Expr()
			match(rparen)
	end

procedure Expr(tokenStream)
	switch(tokenStream.peek())
		case {plus}
			match(plus)
			call Value()
			call Value()
		case {prod}
			match(prod)
			call Values()
	end

procedure Values(tokenStream)
	switch(tokenStream.peek())
		case {num, lparen}
			call Value()
			call Values()
		case {rparen}
			return()
	end
    \end{lstlisting}
    
    
\end{enumerate}

\section{\textit{Dragon}}

\begin{enumerate}

    \item 4.2.1 a \\
    
    S $\Rightarrow$ SS* $\Rightarrow$ SS+S* $\Rightarrow$ aa+a* \\
    
    \item 4.2.1 b \\
    
    S $\Rightarrow$ SS* $\Rightarrow$ Sa* $\Rightarrow$ SS+a* $\Rightarrow$ aa+a*
    
    \item 4.2.1 c \\
    \par
    \Tree [.S [.S [.S a ] [.S a ] + ] [.S a ] * ]
    
    
    
\end{enumerate}

\end{document}

