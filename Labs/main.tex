%%%%%%%%%%%%%%%%%%%%%%%%%%%%%%%%%%%%%%%%%
%
% CMPT 432
% Fall 2019
% Lab 6
%
%%%%%%%%%%%%%%%%%%%%%%%%%%%%%%%%%%%%%%%%%

%%%%%%%%%%%%%%%%%%%%%%%%%%%%%%%%%%%%%%%%%
% Short Sectioned Assignment
% LaTeX Template
% Version 1.0 (5/5/12)
%
% This template has been downloaded from: http://www.LaTeXTemplates.com
% Original author: % Frits Wenneker (http://www.howtotex.com)
% License: CC BY-NC-SA 3.0 (http://creativecommons.org/licenses/by-nc-sa/3.0/)
% Modified by Alan G. Labouseur  - alan@labouseur.com
% Further Modified by Timothy M. Polizzi - Timpolizzi2@gmail.com
% Code Listings by LaTeX
%
%%%%%%%%%%%%%%%%%%%%%%%%%%%%%%%%%%%%%%%%%

%----------------------------------------------------------------
%	PACKAGES AND OTHER DOCUMENT CONFIGURATIONS
%----------------------------------------------------------------

\documentclass[letterpaper, 10pt]{article}

\usepackage[english]{babel} % English language/hyphenation
\usepackage{graphicx}
\usepackage[lined,linesnumbered,commentsnumbered]{algorithm2e}
\usepackage{listings}
\usepackage{fancyhdr} % Custom headers and footers
\pagestyle{fancyplain} % Makes all pages in the document conform to the custom headers and footers
\usepackage{lastpage}
\usepackage{url}
\usepackage{listings}
\usepackage{qtree}
\usepackage{color}
\usepackage{tabto}
\let\epsilon\varepsilon

\definecolor{codegreen}{rgb}{0,0.6,0}
\definecolor{codegray}{rgb}{0.5,0.5,0.5}
\definecolor{codepurple}{rgb}{0.58,0,0.82}
\definecolor{backcolour}{rgb}{0.95,0.95,0.92}

\lstdefinestyle{mystyle}{
    backgroundcolor=\color{backcolour},   
    commentstyle=\color{codegreen},
    keywordstyle=\color{magenta},
    numberstyle=\tiny\color{codegray},
    stringstyle=\color{codepurple},
    basicstyle=\footnotesize,
    breakatwhitespace=false,         
    breaklines=true,                 
    captionpos=b,                    
    keepspaces=true,                 
    numbers=left,                    
    numbersep=5pt,                  
    showspaces=false,                
    showstringspaces=false,
    showtabs=false,                  
    tabsize=2
}

\lstset{style=mystyle}

\fancyhead{} % No page header - if you want one, create it in the same way as the footers below
\fancyfoot[L]{} % Empty left footer
\fancyfoot[C]{page \thepage\ of \pageref{LastPage}} % Page numbering for center footer
\fancyfoot[R]{}

\renewcommand{\headrulewidth}{0pt} % Remove header underlines
\renewcommand{\footrulewidth}{0pt} % Remove footer underlines
\setlength{\headheight}{5pt} % Customize the height of the header

%----------------------------------------------------------------
%	TITLE SECTION
%----------------------------------------------------------------

\newcommand{\horrule}[1]{\rule{\linewidth}{#1}} % Create horizontal rule command with 1 argument of height

\title{	
   \normalfont \normalsize 
   \textsc{CMPT 432 - Fall 2019 - Dr. Labouseur} \\[10pt] % Header stuff.
   \horrule{0.5pt} \\[0.25cm] 	% Top horizontal rule
   \huge Lab Nine\\     	        % Assignment title
   \horrule{0.5pt} \\[0.25cm] 	% Bottom horizontal rule
}

\author{Timothy Polizzi \\ \normalsize Timothy.Polizzi1@Marist.edu}

\date{\normalsize\today} 	% Today's date.

\begin{document}

\maketitle % Print the title

%----------------------------------------------------------------
%   CONTENT SECTION
%----------------------------------------------------------------

\noindent
\section{\textit{Crafting a Compiler}}

\begin{enumerate}
    
    \item 5.5 - \\
    1 DeclList \tab $\rightarrow$ A B \\
    2 A \tab $\rightarrow$ Decl \\
    3 B \tab $\rightarrow$ ; Decl B \\
    4 \tab $|$ $\lambda$ \\
    5 Decl \tab $\rightarrow$ IdList : Type \\
    6 IdList \tab $\rightarrow$ id , IdList \\
    7 \tab $|$ id \\
    8 Type \tab $\rightarrow$ ScalarType \\
    9 \tab $|$ array ( ScalarTypeList ) of Type \\
    10 ScalarType \tab $\rightarrow$ id \\
    11 \tab $|$ Bound . . Bound \\
    12 Bound \tab $\rightarrow$ Sign intconstant \\
    13 \tab $|$ id \\
    14 Sign \tab $\rightarrow$ + \\
    15 \tab $|$ $-$ \\
    16 \tab $|$ $\lambda$ \\
    17 ScalarTypelist \tab $\rightarrow$ ScalarTypeList , ScalarType \\
    18 \tab $|$ ScalarType \\
    
\end{enumerate}

\section{\textit{Dragon}}

\begin{enumerate}

    \item 4.5.3
    \begin{enumerate}
        \item 
        000111 \\
        00S11 \\
        0S1 \\
        S
        
        \item 
        aaa*a++ \\
        aSS*a++ \\
        aSa++ \\
        aSS++ \\
        aS+ \\
        SS+\\
        S
        
        
    \end{enumerate}
    
    
\end{enumerate}

\end{document}